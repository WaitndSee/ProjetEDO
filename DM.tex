\documentclass[10pt]{article}
\usepackage[utf8x]{inputenc}
\usepackage[T1]{fontenc}
\usepackage{graphicx}
\usepackage[french]{babel}
\usepackage{amsmath}


\begin{document}

\title{Rendu du Projet d'EDO	}
\date{January 14, 2016}
\author{Nikita Mauvisseau \and Antoine Weber}
\maketitle
\textbf{Préambule}
~~\\
~~\\

Je suis une brêle en \LaTeX.\\ ainsi qu'en EDO donc ce DM ne va pas être la joie.

Pour l'instant, tout va bie...Arggghhhhh
~~\\

~~\\
\textit{Bip Bip ... Bip ..}
 
~~\\
Entrées de l'inspecteur Dupond le samedi 17 février 2018:\\
 
 ~~\\
11:08  - Nous nous sommes introduit dans les lieux en début de matinée,
 il y règne une atmosphère morbide ,demandons des renforts..
 ~~\\

On se propose aussi d'observer l'augmentation de la masse de l'équipe au cours de la journée,\\
On peut par exemple remarquer que :\\

$$ PoidsMartine = PoidsInitial + PoidsTartine$$


Voici quelques exemples de formules mathématiques. Soit $x$ (et non pas x,
attention) une variable réelle solution de l’équation:
\begin{equation}
ax^2+bx+c=0
\end{equation}
Le discriminant vaut $\Delta=b^2-4ac$. S’il est strictement
positif, il y a deux racines réelles distinctes que l'on écrira plus tard car voilà : flemme
~~\\
~~\\
~~\\
~~\\
~~\\
~~\\


\textbf{Exercice n°4 - Problème d'équation proie-prédateur}
	
texte
	
\end{document}
